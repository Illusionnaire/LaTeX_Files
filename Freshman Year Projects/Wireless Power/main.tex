\documentclass{article}
\usepackage[utf8]{inputenc}

\title{Wireless Power Transfer}
\author{Marcin Wisniowski}
\date{November 9, 2016}

\begin{document}

\maketitle

\tableofcontents

\section{Introduction}
Over the last few decades, the emergence of mobile technology has revolutionized the way people access information and do their jobs. Smart phones have allowed access to the Internet from any corner of the planet, and batteries have grown into a revolutionary movement to store energy for future usage. However, in today’s industries, there is a constant demand for a transportation of energy. As a society that expands and grows by using energy, it is important that constant charging and an empty battery do not limit our success. Imagine a world where your laptop never has to be plugged in, where your phone can charge inside your pocket, or even where cars charge inside your garage. With the increased implementation and research into wireless power, this can soon become the norm. Wireless power techniques allow for the transport of energy over seemingly invisible methods through the use of magnetic waves and electromagnetic fields. Wireless power also allows the transfer of energy through solid objects and across varying distances. In a world connected by a labyrinth of cables, it would be revolutionary to break free. 

\section {History}
Wireless Power is not a new topic of interest and it first grew out of the work of Nikola Tesla. In the era when electricity was being discovered for its benefits, many scientists looked for convenient ways to transport this energy over large distances. Ultimately, many of the wireless techniques including the Wardenclyffe Tower, a large structure devised by Tesla to transfer energy through the atmosphere over a distance of 30 miles, ended up inconsistent and deemed dangerous by the government. Instead, the United States and many European countries implemented an electrical grid of cables to run across the country to supply its powers. Trillions of dollars, and hundreds of years later, these countries are found using the same systems.

\section {Technical Overview}
Energy is transferred with wireless technologies through the use of fields and waves, usually invisible to the human eye. While a steady current produced a fixed magnetic field around it, an alternating field creates a constantly oscillating field. By using receivers that switch these magnetic fields into DC current, wireless power technologies are able to take cabled connections out of the equation. Meanwhile, the vast amount of opportunities for the transfer of wireless energy keeps growing every year. Wireless Communication, which uses signals like radio, Bluetooth and WiFi to transfer information over long distances has energy, no matter how small, stored in it.

\subsection {Power Beaming}
Laser Motive, a company focused on the wireless transfer of energy through lasers, made news in 2010 for using lasers to power a drone to fly without the need to recharge. Laser Motive explored the concept that drives many of our solar panels today, converting light into energy. By focusing a concentrated laser light, energy was able to pass through the air by the use of photon packets. More specifically, by concentrating the laser onto a receiver mounted on the drone, which converted light into DC current, the vehicle was able to stay operational.

\subsection{Electromagnetic Induction Coupling}
Electromagnetic Induction uses completely different methods of energy transfer than power beaming. This method, which is very prominent in the creation of wireless charging stations for phones in today's market, uses magnetic fields and alternating current to transfer energy over distances. An AC current flows through a transmitter in the device which induces a magnetic field around the object. Other objects are then able to be finely tuned, using resonance to match the frequency of the magnetic field and 'couple' with the magnetic field. This creates a bond between the two coils using the magnetic field to dictate how the current would flow through both of them. A strong analogy for this relationship may be apparent when pushing a person on a swing. When the swing is already in motion, it becomes important to time your pushes so that they have a maximum effect on the person swinging. In this same way, the current of the second coil is forced into a pattern that has a maximum effect of keeping the magnetic field strong. Actually, it has been noted that there has been a positive correlation between the strength of a magnetic field and the amount of coils coupled to it. After the second coil is induced into a pattern following that of the first one, a receiver can turn the AC current the coil now has within it into a DC current to power devices with. Already, electromagnetic induction is used in charging batteries and small devices, however there are strong limitations on the maximum distance objects can be apart before the coupling breaks.

\subsection {Wireless Ambient Radio Power}
Wireless Ambient Power takes advantage of many communication waves that are found all around us. The television in a home or the radio in an automobile take in radio waves that are transmitted many kilometers away and convert the information they transfer across those distances into the pictures and sounds that are seen and heard. An Intel Corp research team realized however, that each wave transferred consisted of a small amount of energy. Therefore, they concluded that because of broadcasting towers, there are micro watts available in almost every corner of a large city that can be taped into. By connecting an antenna that received these radio waves to a receiver that converted the energy to DC current, the research team was able to turn on devices, like kitchen thermometers, that only needed a small amount of energy to function. 

\section{Safety Standards}
According to government regulations, all wireless transfer technology adheres to its safety standards and the safety standards of the devices it powers. Likewise, many of the current implementations of wireless power transfer use preexisting fields and waves found on Earth or create them of microscopic scale. One wireless power transfer company, for instance, mentions that the magnetic fields created by the devices are far lower than that of Earth's own magnetic field and therefore completely safe to anyone on Earth. Unfortunately, the large scale implementation of wireless power transfer techniques has its drawbacks. While this technology is completely safe as of now, problems may arise when in the search for stronger energy outputs or longer distances. With the constant push for faster charging and more energy output over long distances, companies are looking to create stronger and larger fields or use more radiative waves in order to increase the energy input of the devices being charged. Microwaves and lasers are only two of many techniques that may grow dangerous if used on a large scale.

\section{Distance Issues}
While wireless energy may have a benefit against cabled energy in that no connection has to be made, it also has the drawback of being unable to focus all of its energy output into one general area. Waves and fields, for instance expand in all directions from their source and get far weaker based on the distance they cover. The fields, and thus the power transmitted, decrease exponentially with the distance away from the source. One way researchers are trying to combat this loss of power is by creating larger coils and efficiently placing distances to maximize the input of energy for the devices. For instance, a coil a foot in diameter would be able to send power efficient over any distance less than a foot before very rapidly losing its efficiency.

\section{Future Technologies}
With the constant push for the implementation and development of wireless technologies, this technology for convenience and automation will hopefully, in the next decade, create revolutionary changes to our everyday lives.

\subsection{Consumer Applications}
Imagine a house where every nook and cranny is powered by wireless technology. You get home from work and through your phone and keys onto the table. The next time you look, they're fully charged and ready to be in use. Within the next decade the world is bound to see a push from single use wireless technology to a strong network of adaptively connected devices. Public parks, restaurants, and airports will all have wireless power available for every person available at any distance. Likewise, there is also bound to be a push for the implementation of automated electric automobile charging. Wireless charging mats are already being made that a car can drive over to charge its battery externally. It is only a matter of time before these limitless ideas become everyday conveniences.

\subsection{Business Applications}
Businesses are also estimated to thrive from the increased benefits of wireless transfer of power. Industries would boom with the increased automation that the lack of charging and battery maintenance would create on a company. Robots would be able to create products without human intervention, until a malfunction occurs in them. Furthermore, environments would no longer prove to be a challenge for electrical needs. The reason the ocean or high altitudes do not have electrical energy are mostly due to the dangerous environments that these cabled connections would have to be built in. By removing the need to create physical connection, research would be able to be made across the globe and even outer space.

\subsection{Governmental Applications}
Government and overall public health is likely to rise due to the increased cabled independence. The convenience created by removing cables and powering all devices in a room for a hospital, for instance would dramatically increase the productivity of its workers. Hospital beds and health monitoring electronics found within the room would become far safer from the issues of contamination. A pacemaker for instance may be made in the future in which it can safely charge itself by simply having the person lay in bed. Lastly, disaster stricken areas, where power outages and road blockades usually occur would be far more navigable with the use of wireless power. Our productivity in all areas of the world would not only increase, but our reliability on the power grid would drastically reduce. 

\section{Conclusion}
Ultimately, while wireless power transfer seems like an issue of convenience for consumers and businesses, the application of this technology can reach out to help any engineering industry, from biomedical to civil to naval. The question now is not if wireless power transfer is possible, but when will it be powerful enough to use reliably.

\begin{thebibliography}{9}
\bibitem{KiYoung} 
Principles and Engineering Explorations edited by Ki Young Kim 
\\\texttt{uv.mx/personal/hvazquez/files/2012/02/InTech-The phenomenon of wireless energy transfer experiments and philosophy.pdf} 
 
\bibitem{2} 
Wireless Power Transfer by Rajeev Mehrotra 
\\\texttt{http://www.cse.wustl.edu/~jain/cse574-14/ftp/power/}

\bibitem{3} 
Wireless Electricity Transmission
\\\texttt{https://www.wirelesspowerconsortium.com/technology/how-it-works.html}

\bibitem{4} 
Power by Proxi
\\\texttt{http://powerbyproxi.com/wireless-power/}

\bibitem{5} 
Tesla’s Tower of Power
\\\texttt{https://www.damninteresting.com/teslas-tower-of-power/}

\bibitem{6} 
Wireless Electricity is Here
\\\texttt{https://www.fastcompany.com/1128055/wireless-electricity-here-seriously}

\bibitem{7} 
Power in the Air
\\\texttt{https://www.youtube.com/watch?v=I1IDC8FEIBU}

\bibitem{8} 
Eric Giler Ted Talk
\\\texttt{https://www.ted.com/talks/eric giler demos wireless electricity?language=en}

\bibitem{9} 
Wikipedia Wireless Power Transfer
\\\texttt{https://en.wikipedia.org/wiki/Wireless power transfer}
\end{thebibliography}


\end{document}
