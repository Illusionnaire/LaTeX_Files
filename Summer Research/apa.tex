\documentclass[jou,apacite]{apa6}

\title{An Agent-Based Model of Consumer Behavior within the Electric Grid}
\shorttitle{APA style}

\twoauthors{Marcin Wisniowski}{Steven Hoffenson}
\twoaffiliations{Stevens Institute of Technology}{Stevens Institute of Technology}

\abstract{\textbf{Abstract:} The goal of this research project is to develop a working simulation to help model the consumption and generation of the U.S. electricity grid. By researching how the electricity market currently works and creating a simulation, we can better understand how each of the parts of the electricity grid interact with each other and help policymakers, both federal and state, to plan and create policies with a stronger impact for their desired purpose.}

\begin{document}
\maketitle    
                        
\section{Introduction}

Carbon dioxide emissions stem many negative consequences on public health and the environment, however they are inevitable in the production of energy and electricity for consumer usage. Renewable policies have come to fruition in order to decrease the negative effects that these sources create. However, while renewables are safer for the environment, they are also more expensive to generate energy. In order to expand the renewable market, consumers must therefore pay more for their energy, either through taxes or by using the renewables (Sundt, Rehdanz, 2014). 

The monthly electricity usage in homes and commercial buildings can very heavily impact the overall energy consumption and the release of greenhouse gases. In New Jersey, over 88\% of electricity ends up being sold to residential and commercial consumers (Wong et al., 2017). Due to this, a consumer's decisions towards energy needs directly impact the total energy usage and carbon dioxide emissions.

By monitoring an individual's energy habits and exploring how they interact with different changes in the energy market, policies can become more effective in converting long-term goals into successful campaigns (Bin, Dowlatabadi, 2005). Improvements in policy can help energy suppliers ease into a greener generating capacity, while not impeding too heavily on profits. An exploration of consumer behavior helps inspect the factors that affect the consumer's actions in energy consumption. To successfully promote renewable energy to the public, the importance of energy prices and energy mix are monitored.

\bibliography{sample}

\end{document}
